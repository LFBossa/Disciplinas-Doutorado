\documentclass{beamer}
\usepackage[utf8]{inputenc}
\usepackage[brazil]{babel}
\usepackage{xcolor}
\usepackage{tikz}
\usetikzlibrary{positioning,calc}
\usepackage{graphicx}
\usepackage{cite}
\usepackage{hyperref}
\usepackage{amsmath}
\usepackage{amssymb}
\usepackage{amsthm}
\usepackage{listings}
\usepackage{fontawesome}
\usetheme{mtmufsc} %%%%%%%%Use this template
\renewcommand{\qedsymbol}{$\blacksquare$}
% This is a beamer template inspired by unofficial Oxford University Beamer Template, made by Clara Eleonore Pavillet.
\title{Princípio do Máximo e EDPs Elípticas}
\author{Luiz Fernando Bossa}
\date{7 de novembro de 2025}
\institute{Universidade Federal de Santa Catarina}
 
\newcommand{\RR}{\mathbb{R}} 
\newcommand{\dR}{\partial R}
\newcommand{\simpleitem}[1]{
    \begin{itemize}
        \item #1
    \end{itemize}}

\begin{document}

\frame{\titlepage}
% https://arxiv.org/pdf/2104.13478

\section{EDPs Elípticas}

\begin{frame}{Formulação do Problema}
\begin{itemize}
    \item  Seja $R \subset \RR^n$ um domínio limitado com fronteira $\dR$ suave. 
    \item  Considere a equação diferencial parcial
    \end{itemize} \medskip
    {\small\begin{equation*}\tag{4.1}
        a(x,y)\frac{\partial^2 u}{\partial x^2} + 2b(x,y)\frac{\partial^2 u}{\partial x \partial y} + c(x,y)\frac{\partial^2 u}{\partial y^2}  = d(x,y,u,u_x,u_y)
    \end{equation*}}
    \begin{itemize}
     \item     A equação é dita elíptica em $R$ se $b^2 - ac < 0$ em $R$.
\end{itemize}
\end{frame}

\begin{frame}{Notação e convenções}

\begin{itemize}[<+->]
    \item $R = [0,a]\times[0,b]$;
    \item $x_i = i\cdot h$, $y_j = j\cdot k$;
    \item $R_\delta$ e $\dR_\delta$ são as versões discretizadas do interior e da borda de $R$;
    \item $U_{i,j}$ é a aproximação da solução em $(x_i,y_j)$  
    \item O laplaciano discreto é dado por
    \[\Delta_\delta U_{i,j} := \frac{U_{i+1,j}-2U_{i,j} + U_{i-1,j}}{h^2} +  \frac{U_{i,j+1}-2U_{i,j} + U_{i,j-1}}{k^2}  \]
\end{itemize}
\end{frame}

\begin{frame}{Princípio do Máximo}

\begin{block}{Teorema 4.1 (Princípio do Máximo)}
    Seja $V(x,y)$ uma função discreta (de malha) definida sobre $R_\delta\cup\dR_\delta$.
    \begin{enumerate}[(a)]
        \item Se $\Delta_\delta V \ge 0$ em $R_\delta$, então 
        \[\max_{(x,y)\in R_\delta} V(x,y) \le \max_{(x,y)\in \dR_\delta} V(x,y),\]
        isto é, o máximo de $V$ acontece na borda.
        \item Se $\Delta_\delta V \le 0$ em $R_\delta$,
        \[\min_{(x,y)\in R_\delta} V(x,y) \ge \min_{(x,y)\in \dR_\delta} V(x,y),\]
        isto é, o mínimo de $V$ acontece na borda.
    \end{enumerate}
\end{block}
    
\end{frame}

\begin{frame}{Demonstração}
\begin{itemize}[<+->]
    \item Demonstração por contradição para o caso (a).
    \item Suponha que o máximo geral acontece em \[P_0 := (x_r,y_s) \in R_\delta.\]
    \item Denote por $M_0$ o valor de $V(P_0)$.
    \item Temos que 
    \begin{align*}
        M_0 &\ge V(P) & \forall P\in R_\delta \\
        M_0 &> V(P) &  \forall P\in\dR_\delta \label{HP}\tag{H.P}
    \end{align*} 
\end{itemize}
    
\end{frame}

\begin{frame}{Demonstração}
    \begin{columns}[T]
        \column{0.55\textwidth}{

            \begin{itemize}
                \item Defina
                \begin{align*}
                    P_1 &= (x_{r+1},y_s)\\
                    P_2 &= (x_{r-1},y_s)\\
                    P_3 &= (x_{r},y_{s+1})\\
                    P_4 &= (x_{r},y_{s-1})\\
                \end{align*}
                e veja que podemos escrever 
            \end{itemize}}
        \column{0.4\textwidth}{
\pause
        \begin{tikzpicture}
        \draw[gray!45,dashed] (-1.8,-1.8) grid (1.8,1.8);
            % 1. Define todos os pontos como coordenadas
            \coordinate (P0) at (0,0);
            \coordinate (P1) at (1,0);
            \coordinate (P2) at (-1,0);
            \coordinate (P3) at (0,1);
            \coordinate (P4) at (0,-1);
        
            % 2. Adiciona os labels (nodes) para fora de cada coordenada
            \node[above right] at (P0) {$P_0$};
            \node[right,fill=white] at (P1) {$P_1$};
            \node[left,fill=white] at (P2) {$P_2$};
            \node[above,fill=white] at (P3) {$P_3$};
            \node[below,fill=white] at (P4) {$P_4$};
        
            % Desenha as linhas que formam os eixos
            \draw (P1) -- (P2) (P3) -- (P4);
        
            % 3. Coloca um ponto de 1pt de diâmetro (0.5pt de raio) em cada coordenada
            % Usando um loop para ser mais eficiente
            \foreach \i in {0,1,2,3,4} {
                \draw[fill=black] (P\i) circle (1pt);
            }
        \end{tikzpicture} }
\end{columns}\pause\small
\[
\Delta_\delta V(P_0) = \frac{V(P_1)+V(P_2)}{h^2}+\frac{V(P_3)+V(P_4)}{k^2} - 2\left(\frac{1}{h^2}+\frac{1}{k^2}\right)V(P_0)
\]
\end{frame}

\begin{frame}{Demonstração}
    \simpleitem{Isolando tudo o que depende de $P_0$ temos}\small\pause
    \[
2\left(\frac{1}{h^2}+\frac{1}{k^2}\right)V(P_0) + \Delta_\delta V(P_0) = \frac{V(P_1)+V(P_2)}{h^2}+\frac{V(P_3)+V(P_4)}{k^2} 
\] \normalsize\pause
\simpleitem{Como $\Delta_\delta V(P_0) \ge 0$, temos}
\small
\[
2\left(\frac{1}{h^2}+\frac{1}{k^2}\right)V(P_0)  \le \frac{V(P_1)+V(P_2)}{h^2}+\frac{V(P_3)+V(P_4)}{k^2} 
\]\normalsize\pause
\simpleitem{Isolando $V(P_0) = M_0$}\footnotesize
\begin{equation}
M_0 \le  \left(\frac{1}{h^2}+\frac{1}{k^2}\right)^{-1}\left(\frac1{h^2}\frac{V(P_1)+V(P_2)}{2}+\frac1{k^2}\frac{V(P_3)+V(P_4)}{2} \right)
\tag{4.19}
\end{equation}
\end{frame}


 \begin{frame}{Demonstração}
 \simpleitem{Como $M_0$ é o máximo no interior (e por hipótese também na borda), em particular:}
 \begin{equation*}\tag{$\star$}\label{estrela}
     V(P_i) \le  M_0  \quad i=1,\ldots,4 
 \end{equation*} 
 Assim
 \small
 \[
 \frac{V(P_1)+V(P_2)}{2} \le M_0 \quad \text{e} \quad \frac{V(P_3)+V(P_4)}{2} \le M_0
 \]\pause
\normalsize
 \simpleitem{Supondo que a desigualdade é estrita para algum $i$, então uma das desigualdades acima também vira estrita e vale:}\small
\begin{multline*}
\frac1{h^2}\frac{V(P_1)+V(P_2)}{2}+\frac1{k^2}\frac{V(P_3)+V(P_4)}{2} < \\ < \frac1{h^2}M_0+\frac1{k^2}M_0 = \left(\frac1{h^2}+\frac1{k^2} \right)M_0
\end{multline*}
\end{frame}

\begin{frame}{Demonstração}
\simpleitem{Substituindo a desigualdade estrita em (4.19)}
\small
\begin{multline*}
M_0 \le  \left(\frac{1}{h^2}+\frac{1}{k^2}\right)^{-1}\left(\frac1{h^2}\frac{V(P_1)+V(P_2)}{2}+\frac1{k^2}\frac{V(P_3)+V(P_4)}{2} \right) <  \\ <
\left(\frac{1}{h^2}+\frac{1}{k^2}\right)^{-1}\left(\frac1{h^2}+\frac1{k^2} \right)M_0 = M_0
\end{multline*}\normalsize
chegamos em uma contradição $M_0 < M_0$.\pause
\simpleitem{Logo, em \eqref{estrela} só pode valer a igualdade, e para  todo $i=1,\ldots,4$ tem que valer $V(P_i) = M_0$. }
\end{frame}

\begin{frame}{Demonstração}
    \simpleitem{Agora basta repetir o argumento para cada um dos $P_i$}
    \centering
    \begin{tikzpicture}
        \draw[gray!45,dashed] (-1.8,-1.8) grid (2.5,2.5);
            % 1. Define todos os pontos como coordenadas
            \coordinate (P0) at (0,0);
            \coordinate (P1) at (1,0);
            \coordinate (P2) at (-1,0);
            \coordinate (P3) at (0,1);
            \coordinate (P4) at (0,-1);
            \coordinate (P5) at (2,0);
            \coordinate (P6) at (1,1);
            \coordinate (P7) at (1,-1);
        
            % 2. Adiciona os labels (nodes) para fora de cada coordenada
            \onslide<1>{
            \node[above right] at (P0) {$P_0$};
            }
            \onslide<2->{
            \node[above right] at (P0) {$M_0$};
            }
            \onslide<2>{
            \node[right,fill=white] at (P1) {$P_1$};
            \node[left,fill=white] at (P2) {$P_2$};
            \node[above,fill=white] at (P3) {$P_3$};
            \node[below,fill=white] at (P4) {$P_4$};
            }
            \onslide<3->{
            \node[above right,fill=white] at (P1) {$M_0$};
            \node[left,fill=white] at (P2) {$M_0$};
            \node[above,fill=white] at (P3) {$M_0$};
            \node[below,fill=white] at (P4) {$M_0$};
            }
            % Desenha as linhas que formam os eixos
            \draw (P1) -- (P2) (P3) -- (P4);
        
            % 3. Coloca um ponto de 1pt de diâmetro (0.5pt de raio) em cada coordenada
            % Usando um loop para ser mais eficiente
            \foreach \i in {0,1,2,3,4} {
                \draw[fill=black] (P\i) circle (1pt);
            }
            \onslide<4->{
            \foreach \i in {5,6,7} {
                \draw[fill=black] (P\i) circle (1pt);
                }
                
            \draw (P1) -- (P5) (P6) -- (P7);
            }
            \onslide<5->{
            \node[right,fill=white] at (P5) {$M_0$};
            \node[above,fill=white] at (P6) {$M_0$};
            \node[below,fill=white] at (P7) {$M_0$};
            }
        \end{tikzpicture}
\end{frame}

\begin{frame}{Demonstração}
\begin{itemize} 
    \item Chegamos assim a conclusão de que $V\equiv M_0$ em todo o domínio, incluindo a borda.
    \item Mas isso entra em contradição com a hipótese principal \eqref{HP}, de que $M_0$ é estritamente maior que os valores na borda.
    \item Segue que o máximo da função deve acontecer na borda. 
\end{itemize}
\end{frame}

\begin{frame}{Demonstração}
\begin{itemize}[<+->]
    \item Suponha que $V$ satisfaz as hipóteses de (b), defina $G := -V$
    \item Note que vale
    \[
        \Delta_\delta G =  \Delta_\delta (-V) =  -\Delta_\delta V \ge 0
    \]
    \item Logo, por (a), temos 
    \[
      \max_{R_\delta} G(x,y) \le \max_{\dR_\delta} G(x,y)
    \]
    \item Como $\max G = \max -V = -\min V$, temos 
    \[
    -\min_{R_\delta} V(x,y) \le -\min_{\dR_\delta} V(x,y)
    \]
    e segue (b).
\end{itemize}
\end{frame}

\begin{frame}{Teorema}
\begin{block}{Teorema 4.2}
Seja $V(x,y)$ uma função discreta (de malha) definida sobre $R_\delta\cup\dR_\delta$. Então
\begin{equation*}\tag{4.20}
\max_{R_\delta} |V(x,y)| \le \max_{\dR_\delta} |V(x,y)| + \frac{a^2}{2}\max_{\dR_\delta}|\Delta_\delta V(x,y)|    
\end{equation*}
\end{block}
\end{frame}

\begin{frame}{Demonstração}
    \begin{itemize}[<+->]
        \item Considere $\phi(x,y) = x^2/2$
        \item Veja que dentro do domínio,
        \[0\le \phi(x,y)\le a^2/2\quad\text{e}\quad\Delta_\delta \phi(x,y) = 1\]
        \item Seja $N_0 := \max_{R_\delta} |\Delta_\delta V(x,y)|$
        \item Defina $V_+$ e $V_-$ por 
        \[
        V_\pm(x,y) := \pm V(x,y) + N_0\phi(x,y)
        \]
    \end{itemize}
\end{frame}

\begin{frame}{Demonstração}
    \begin{itemize}[<+->]
    \item Calculando o laplaciano e usando a linearidade
    \begin{multline*}
    \Delta_\delta V_\pm(x,y) = \Delta_\delta\big( \pm V(x,y) + N_0\phi(x,y)\big) = \\ = \pm \Delta_\delta V(x,y) + N_0\Delta_\delta\phi(x,y) =    \pm \Delta_\delta V(x,y) + N_0 
    \end{multline*} 
    \item Para qualquer $\zeta\in\RR$ vale $\mp\zeta \le |\zeta|$, logo pela definição de $N_0$
    \begin{align*}
        \mp \Delta_\delta V(x,y) \le |\Delta_\delta V(x,y)| &\le N_0 \\
        \mp  \Delta_\delta V(x,y) \pm  \Delta_\delta V(x,y) &\le N_0 \pm  \Delta_\delta V(x,y)\\
        0 &\le  \Delta_\delta V(x,y) + N_0
    \end{align*}
    \end{itemize}    
\end{frame}
\begin{frame}{Demonstração}
    \begin{itemize}
        \item Aplicando o Princípio do Máximo para $V_\pm$, temos que o máximo acontece na borda, ou seja
        \begin{multline*}\tag{$\dagger$}
        V_\pm(x,y) \le \max_{\dR_\delta} V_\pm(x,y) =  \max_{\dR_\delta} \big( \pm V(x,y) + N_0\phi(x,y)\big) \\ \le  \max_{\dR_\delta} \big(\pm V(x,y)\big) + N_0\frac{a^2}{2} \\ \le  \max_{\dR_\delta} \big|V(x,y)\big| + N_0\frac{a^2}{2}               
        \end{multline*}
        pois $\pm V \le |V|$
    \end{itemize}
\end{frame}

\begin{frame}{Demonstração}
    \begin{itemize}[<+->]
        \item Note que $N_0, \phi \ge 0$, logo 
        \[
        \pm V(x,y) \le  \pm V(x,y) +N_0\phi(x,y) = V_\pm(x,y)
        \]
        \item Colando a desigualdade acima com $(\dagger)$ temos
        \[
        \pm V(x,y) \le  \max_{\dR_\delta} \big| V(x,y)\big| + N_0\frac{a^2}{2}  
        \]
        \item O lado direito não depende de $R_\delta$, então podemos tomar o máximo do lado esquerdo sobre o interior do domínio, segue o teorema.
    \end{itemize}
\end{frame}

\begin{frame}{Colorário}
\begin{block}{Corolário 4.1}
        O sistema de equações lineares formado pela discretização da EDP elípica
        \[
        AU = r
        \]
        tem uma única solução. 
\end{block}
    
\end{frame}

\begin{frame}{Demonstração}
\begin{itemize}[<+->]
    \item Vamos provar que $AU = 0$ tem somente a solução zero, o que prova $A$ injetora. Como $A$ é quadrada, injetora implica inversível. 
    \item Considere o sistema homogêneo
    \begin{align}
        -\Delta_\delta U_{i,j} &=0 & \text{em } R_\delta \tag{4.9} \\
        U_{i,j} &= 0 & \text{em } \dR_\delta \tag{4.10} 
    \end{align}
\end{itemize}
\end{frame}

\begin{frame}{Demonstração 1}
    \begin{itemize}[<+->]
        \item Aplicando o Teorema 4.2 para $U_{i,j}$ temos
        \[
        \max_{R_\delta} |U_{i,j}| \le \underbrace{\max_{\dR_\delta} |U_{i,j}|}_{(4.10)} + \frac{a^2}{2}\underbrace{\max_{R_\delta} |\Delta_\delta U_{i,j}|}_{(4.9)} = 0 + \frac{a^2}{2}0 = 0
        \]
        \item Segue que $|U_{i,j}| = 0$ e logo $U_{i,j} = 0$.
    \end{itemize}
\end{frame}

\begin{frame}{Demonstração 2}
    \begin{itemize}[<+->]
        \item Sabemos que  $\Delta_\delta U_{i,j} = 0$ em $R_\delta$;
        \item Aplicando o Princípio do Máximo, podemos usar tanto (a) quanto (b), logo o máximo e o mínimo de $U_{i,j}$ acontecem na borda;
        \item Mas isso significa que $\max U_{i,j} = \min U_{i,j} = 0$, de modo que $U$ é zero.
    \end{itemize}
\end{frame}
\end{document}


