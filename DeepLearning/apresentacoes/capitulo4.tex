\documentclass{beamer}
\usepackage[utf8]{inputenc}
\usepackage[brazil]{babel}
\usepackage{xcolor}
\usepackage{tikz}
\usetikzlibrary{positioning,calc}
\usepackage{graphicx}
\usepackage{cite}
\usepackage{hyperref}
\usepackage{amsmath}
\usepackage{amssymb}
\usepackage{amsthm}
\usepackage{listings}
\usepackage{enumerate}
\usepackage{fontawesome}
\usepackage{ulem}
\usepackage{xfrac}
\usepackage{oands}
\usepackage{hieroglf}


%https://github.com/battlesnake/neural
%
\usepackage{neuralnetwork}
%https://pt.overleaf.com/latex/templates/template-departamento-de-mtm-ufsc/jxjxqttwjrws
\usetheme{mtmufsc} %%%%%%%%Use this template
\renewcommand{\qedsymbol}{$\blacksquare$}
% This is a beamer template inspired by unofficial Oxford University Beamer Template, made by Clara Eleonore Pavillet.
\title{RG Flow of Preactivations}
\author{Luiz Fernando Bossa}
\date{\today}
\institute{Universidade Federal de Santa Catarina}


\newcommand{\PP}{\mathbb{P}}
\newcommand{\EE}{\mathbb{E}}
\newcommand{\RR}{\mathbb{R}}
\DeclareMathOperator{\Cov}{Cov}
\DeclareMathOperator{\Var}{Var}
\DeclareMathOperator{\Wick}{Wick}
\newcommand{\vX}{\vec{X}}
\newcommand{\vY}{\vec{Y}}
\newcommand{\vmu}{\vec{\mu}}
\newcommand{\WW}{\mathcal{W}}
\newcommand{\aaA}{\alpha}
\newcommand{\aaB}{\beta} 
\newcommand{\OO}{\mathcal{O}}
\newcommand{\Dcal}{\mathcal{D}}
\newcommand{\Ical}{\mathcal{I}}
\newcommand{\Ncal}{\mathcal{N}}
\newcommand{\Gcal}{\mathcal{G}}
\newcommand{\Acal}{\mathcal{A}}
\def\mi#1{{\alpha_{#1}}}
\def\mj#1{\beta_{#1}}
\def\eell{{(\ell)}}
\def\eellum{{(\ell+1)}}
\def\wickquatro{\delta_{\mi1\mi2}\delta_{\mi3\mi4} + \delta_{\mi1\mi3}\delta_{\mi2\mi4}+  \delta_{\mi1\mi4}\delta_{\mi2\mi3}}
\def\ddelta#1#2{\delta_{\mi{#1}\mi{#2}}}
\def\Gchapp#1{\widehat{G}^{(#1)}}
%\def\Gchapeu#1#2{\widehat{G}^{(#1)}_{\alpha_{#2}\alpha_{#3}}} 
\newcommand{\Gchapeu}[3]{{\Gchapp{#1}_{\mi{#2}\mi{#3}}}} 
\newcommand{\Gchapeuinv}[3]{\widehat{G}_{(#1)}^{\mi{#2}\mi{#3}}}
\newcommand{\Gnormal}[3]{{G^{(#1)}_{\mi{#2}\mi{#3}}}} 
\newcommand{\Gnormalinv}[3]{{G_{(#1)}^{\mi{#2}\mi{#3}}}} 
\newcommand{\Gflutu}[3]{{\widehat{\Delta G}^{(#1)}_{\mi{#2}\mi{#3}}}} 
\newcommand{\Vertice}[5]{V^{(#1)}_{(\mi#2\mi#3)(\mi#4\mi#5)}}
\newcommand{\Verticeb}[5]{V^{(#1)}_{(\mj#2\mj#3)(\mj#4\mj#5)}}
\newcommand{\Verticeinv}[5]{V_{(#1)}^{(\mi#2\mi#3)(\mi#4\mi#5)}}
\newcommand{\Verticeinvb}[5]{V_{(#1)}^{(\mj#2\mj#3)(\mj#4\mj#5)}}
\newcommand{\Expectation}[2]{\left\langle #1 \right\rangle_{#2}}
\newcommand{\zia}[2]{z_{i_{#1};\mi{#2}}}



\begin{document}

{\setbeamertemplate{footline}{} 
\frame{\titlepage}}
\frame{\tableofcontents}


\section{Recap}    
%Sempre que iniciar uma nova sessão, você pode fazer um slide de transição com o índice.
\begin{frame}
\tableofcontents[currentsection]
\end{frame}

%%%%%%%%%%%%%%%%%%%%%%%%%%%%%%%%%%%%%%%%%%% 
%A partir daqui, faça seus slides%%%%%%%%%%%%%%%%%%%%%%%%%%%%

\begin{frame}{\S 4.2 Second Layer: Genesis of Non-Gaussianity}
	\begin{itemize}
		\item Cálculo da distribuição condicional
		\begin{equation*}\tag{4.32}
			p\left(z^{(2)},z^{(1)}\Big|\Dcal\right) = p\left(z^{(2)}\Big| z^{(1)}\right)p\left(z^{(1)}\Big| \Dcal\right)
		\end{equation*}
		\begin{equation*}\tag{4.35}
			p\left(z^{(2)}\Big| z^{(1)}\right) = \frac{1}{\sqrt{\left|2\pi \hat{G}^{(2)}\right|^{n_2}}}
			\exp\left(-\frac{1}{2} \sum_{\mi1,\mi2\in\Dcal} \Gchapeuinv212 z^{(2)}_{\mi1}\cdot z^{(2)}_{\mi2} \right)
		\end{equation*}
	\end{itemize}
\end{frame}

\begin{frame}{\S 4.2 Second Layer: Genesis of Non-Gaussianity}

\begin{itemize}
	\item 
	Métrica estocástica da 2ª camada
	\begin{equation*}\tag{4.36}
		\Gchapeu212 := C_b^{(2)} + C_W^{(2)}\frac{1}{n_1}\sum_{j=1}^{n_1} \sigma_{j;\mi1}^{(1)}\sigma_{j;\mi2}^{(1)}
	\end{equation*}

	\item Média da métrica da 2ª camada
	\begin{align*}
		\Gnormal212 :&= \EE\left[\Gchapeu212\right] 
		= C_b^{(2)} + C_W^{(2)}\frac{1}{n_1}\sum_{j=1}^{n_1} \EE\left[\sigma_{j;\mi1}^{(1)}\sigma_{j;\mi2}^{(1)}\right]\\
		&= C_b^{(2)} + C_W^{(2)}\Expectation{\sigma_{\mi1}\sigma_{\mi2}}{G^{(1)}}\tag{4.37}
	\end{align*}
\end{itemize}

\end{frame}

\begin{frame}{\S 4.2 Second Layer: Genesis of Non-Gaussianity}
\begin{itemize}
	\item Flutuação da 2ª camada: desvio da média
	\begin{equation*}\tag{4.38}
		\Gflutu212 := \Gchapeu212 - \Gnormal212
	\end{equation*}
	\item Vértice de 4 pontos: tamanho médio da flutuação
	\begin{multline*}
		\EE\left[\Gchapeu212\Gchapeu234\right]  = \\ \frac{1}{n_1}\big(C_W^{(2)}\big)^2\big(  \Expectation{\sigma_\mi1\sigma_\mi2\sigma_\mi3\sigma_\mi4}{G^{(1)}} -  \Expectation{\sigma_\mi1\sigma_\mi2 }{G^{(1)}}  \Expectation{\sigma_\mi3\sigma_\mi4}{G^{(1)}}\big) \\
		=: \Vertice21234 \tag{4.40}
	\end{multline*}
\end{itemize}
\end{frame}


\section{Deeper Layers: Accumulation of Non-Gaussianity}

\begin{frame}
\tableofcontents[currentsection]
\end{frame}

\begin{frame}
	\begin{itemize}
		\item Pré-ativação na camada $\ell+1$ é dada por
	\end{itemize}
	\begin{equation*}
		z_{i;\alpha}^\eellum = b_i^\eellum + \sum_{j=1}^{n_\ell} W_{ij}^\eellum\sigma_{j;\alpha}^\eell
	\end{equation*}
	com 
	$$\sigma_{j;\alpha}^\eell := \sigma\left(z_{i;\alpha}^\eell\right)$$
\end{frame}

\subsection{Recursion}

\begin{frame}
	\begin{equation*}\tag{4.67}
		p\left(z^\eellum,z^\eell\Big|\Dcal\right) = p\left(z^\eellum\Big| z^\eell\right)p\left(z^\eell\Big| \Dcal\right)
	\end{equation*}
	Distribuição condicional camada $\ell+1$
{\footnotesize
	\begin{equation*}\tag{4.69}
			p\left(z^\eellum\Big| z^\eell\right) = \frac{1}{\sqrt{\left|2\pi \hat{G}^\eellum\right|^{n_{\ell+1}}}}
			\exp\left(-\frac{1}{2} \sum_{\mi1,\mi2\in\Dcal} \Gchapeuinv{\ell+1}12 z^\eellum_{\mi1}\cdot z^\eellum_{\mi2} \right)
		\end{equation*}
}
Métrica estocástica da camada $\ell+1$
	\begin{equation*}\tag{4.70}
		\Gchapeu{\ell+1}12 := C_b^\eellum + C_W^\eellum\frac{1}{n_1}\sum_{j=1}^{n_1} \sigma_{j;\mi1}^\eell\sigma_{j;\mi2}^\eell
	\end{equation*}
\end{frame}

\begin{frame}
	Média da métrica estocástica da camada $\ell+1$
	\begin{equation*}\tag{4.72}
		\Gnormal{\ell+1}12 := \EE\left[\Gchapeu{\ell+1}12 \right] =  C_b^\eellum + C_W^\eellum\frac{1}{n_1}\sum_{j=1}^{n_\ell} \EE\left[\sigma_{j;\mi1}^\eell\sigma_{j;\mi2}^\eell\right]
	\end{equation*}

	Essa média governa o correlator de dois pontos

	\begin{equation*}\tag{4.73}
		\EE\left[\zia11^\eellum \zia22^\eellum\right] = \delta_{i_1i_2}\Gnormal{\ell+1}12 
	\end{equation*}
\end{frame}

\begin{frame}
	
	Flutuação da métrica 
	\begin{equation*}\tag{4.74}
		\Gflutu{\ell+1}12 := \Gchapeu{\ell+1}12 - \Gnormal{\ell+1}12
	\end{equation*}
	Magnitude da flutuação
	\begin{equation*}\tag{4.76}
		\frac{1}{n_\ell}\Vertice{\ell+1}1234 := \EE\left[\Gflutu{\ell+1}12 \Gflutu{\ell+1}34 \right]
	\end{equation*}
\end{frame}

\begin{frame}
	\begin{multline*}
		\EE\left[\zia11^\eellum \zia22^\eellum \zia33^\eellum \zia44^\eellum\right]\Big|_{C} =\\= \frac{1}{n_\ell}\left( 
			\delta_{i_1i_2}\delta_{i_3i_4} \Vertice{\ell+1}1234 +
			\delta_{i_1i_3}\delta_{i_2i_4} \Vertice{\ell+1}1324 +\right.\\+\left.
			\delta_{i_1i_4}\delta_{i_2i_3} \Vertice{\ell+1}1423 
		\right)\tag{4.77}
	\end{multline*}
\end{frame}

\subsection{Action}

\begin{frame}
	Podemos definir a distribuição na camada $\ell$ através da ação
	\begin{equation*}\tag{4.78}
		p\left(z^\eell \Big| \Dcal\right) = \frac{e^{-S(z^\eell)}}{Z_\ell}
	\end{equation*}
	com 
	\begin{equation*}\tag{4.79}
		Z_\ell := \int\bigg[{\prod_{i,\alpha}} dz_{i;\alpha}^\eell\bigg]e^{-S(z^\eell)}
	\end{equation*}
	sendo o termo de normalização.
\end{frame}

\begin{frame}
	Nosso modelo para a ação $S$ será
	{\footnotesize
	\begin{equation*}\tag{4.80}
		S(z^\eell) := \frac{1}{2}\sum_{\mi1,\mi2}g^{\mi1\mi2}_{\eell} z_{\mi1}{\cdot}z_{\mi2} - \frac{1}{8}\sum_{\mi{i}\in\Dcal}^{1\le i \le 4} v^{(\mi1\mi2)(\mi3\mi4)}_{\eell} z_{\mi1}{\cdot}z_{\mi2} z_{\mi3}{\cdot}z_{\mi4}+\ldots
	\end{equation*}}
	\begin{itemize}
		\item Esse modelo funciona para a camada 1 com 
		$$g^{\mi1\mi2}_{(1)} = \Gnormalinv112, \qquad v_{(1)}^{(\mi1\mi2)(\mi3\mi4)} = 0.$$
		\item Funciona para a camada 2 com 
		{\small
		$$g^{\mi1\mi2}_{(2)} = \Gnormalinv212 + {O}\left(\sfrac{1}{n_1}\right), \quad v_{(2)}^{(\mi1\mi2)(\mi3\mi4)} = \frac{1}{n_1}\Verticeinv21234 + O\left(\sfrac{1}{n_1^2}\right) $$ }
	\end{itemize}
\end{frame}
\begin{frame}
	
Por analogia, temos
	\begin{equation*}\tag{4.81}
		g^{\mi1\mi2}_{\eell} = \Gnormalinv{\ell}12 + \mathcal{O}(v)
	\end{equation*}
	e 
	\begin{equation*}\tag{4.82}
		v^{(\mi1\mi2)(\mi3\mi4)}_{\eell} = \frac{1}{n_{\ell-1}}\Verticeinv{\ell}1234 + \mathcal{O}(v^2)
	\end{equation*}
no qual o vértice invertido é dado por
	\begin{equation*}\tag{4.83}
		\Verticeinv{\ell}1234 := \sum_{\mj{i}\in\Dcal}^{1\le i \le 4} G^{\mi1\mj1}_\eell G^{\mi2\mj2}_\eell G^{\mi3\mj3}_\eell G^{\mi4\mj4}_\eell V^\eell_{(\mi1\mi2)(\mi3\mi4)}
	\end{equation*}
\end{frame}

\subsection{Large-width expansion}

\begin{frame}{Large-width expansion}
	\begin{itemize}
		\item Simplificamos os cálculos fazendo 
		\begin{equation*}\tag{4.84}
			n_1, n_2, \ldots, n_L \sim n \gg 1 
		\end{equation*} 
	\end{itemize}
\end{frame}

\begin{frame}{Indução}
	\begin{block}{Teorema}
		Se as métricas $G^\eell$ e $V^\eell$ são de ordem de grandeza constante $O(1)$, então $G^\eellum$ e $V^\eellum$ também são de ordem de grandeza constante.
	\end{block}
\end{frame}

\begin{frame}
	Pela equação (4.72), temos que a métrica $G$ da camada $\ell+1$ é dada por
	\begin{equation*}
		\Gnormal{\ell+1}12 = C_b^\eellum + C_W^\eellum\frac{1}{n_\ell}\sum_{j=1}^{n_\ell}  \EE\left[\sigma_{j;\mi1}^\eell\sigma_{j;\mi2}^\eell\right]
	\end{equation*}

	Na sessão anterior, vimos a expressão para a esperança dentro do somatório.
\end{frame}

\begin{frame}
	A equação (4.61) calculada na sessão anterior nos dá a terrível fórmula 
	{\small
	\begin{multline*}
		\EE\left[\sigma_{j;\mi1}^\eell\sigma_{j;\mi2}^\eell\right] = \Expectation{\sigma_\mi1\sigma_\mi2}{G^\eell}  + \frac{1}{8} \sum_{\mj{i}\in\Dcal}^{1\le i \le 4}v_{\eell}^{(\mj1\mj2)(\mj3\mj4)}\Big( \text{\pmglyph{Y}} \Big) + O(v^2) =\\
		=  \Expectation{\sigma_\mi1\sigma_\mi2}{G^\eell} { \color{blue} + \frac{1}{8} \sum_{\mj{i}\in\Dcal}^{1\le i \le 4}\frac{1}{n_\ell}\Verticeinvb{\ell}1234\Big( \text{\pmglyph{Y}} \Big) } + O(\sfrac{1}{n_\ell^2}) = \\
	%	= \Expectation{\sigma_\mi1\sigma_\mi2}{G^\eell} + \text{\pmglyph{Q}}  + O(\sfrac{1}{n_\ell^2})
	\end{multline*}
	}
	em que o hieróglifo \pmglyph{Y} representa a exata sensação ao ver essa expressão. 
\small 
\begin{multline*} \text{\pmglyph{Y}} = \Expectation{\sigma_{\mi1} \sigma_{\mi2} ( z_{\mj1}z_{\mj2} - g_{\mj1 \mj2} ) ( z_{\mj3} z_{\mj4} - g_{\mj3 \mj4}) }{g} \\
 + 2n \Expectation{\sigma_{\mi1} \sigma_{\mi2} ( z_{\mj1} z_{\mj2} - g_{\mj1 \mj2}) }{g} g_{\mj3 \mj4} - 2 \Expectation{ \sigma_{\mi1} \sigma_{\mi2} }{g} g_{\mj1 \mj3} g_{\mj2 \mj4} \end{multline*}
\end{frame}

\begin{frame}
	\begin{itemize}
		\item  \pmglyph{Y} tem um termo de ordem $n_\ell$, que se torna de ordem constante quando dividimos por $n_\ell$. 
		\item Esse termo de ordem constante vamos chamar de \pmglyph{Q}.
	\end{itemize}
	\begin{equation*}
		\EE\left[\sigma_{j;\mi1}^\eell\sigma_{j;\mi2}^\eell\right] = \Expectation{\sigma_\mi1\sigma_\mi2}{G^\eell} + \text{\pmglyph{Q}} + O(\sfrac{1}{n_\ell}) +  O(\sfrac{1}{n_\ell^2})
	\end{equation*}
\end{frame}


\begin{frame}
	Assim, a métrica da camada $\ell+1$ é dada por\footnote{Isso segundo os cara, eu acho que falta uma sujeirinha \pmglyph{Q} aqui.\\\vspace*{18pt}}
	\begin{multline*} 
		\Gnormal{\ell+1}12 = C_b^\eellum + C_W^\eellum\frac{1}{n_\ell}\sum_{j=1}^{n_\ell} \bigg[\Expectation{\sigma_\mi1\sigma_\mi2}{G^\eell} + O(\sfrac{1}{n}) \bigg] \\
		= C_b^\eellum + C_W^\eellum{\color{red}\Expectation{\sigma_\mi1\sigma_\mi2}{G^\eell} } + O\left(\sfrac{1}{n}\right)
	\end{multline*}
	Pela hipótese de indução, essa expectativa em vermelho é de ordem constante. Segue que a métrica da camada $\ell+1$ é de ordem constante.
\end{frame}

\begin{frame}
	Para o vértice de quatro pontos, temos
	{\small
	\begin{multline*}
		\frac{1}{n_\ell}\Vertice{\ell+1}1234 =  \left(\frac{C_W^\eellum}{n_\ell}\right)^2 \sum_{j,k=1}^{n_\ell}  \left\{ \EE\left[\sigma_{j;\mi1}^\eell\sigma_{j;\mi2}^\eell \sigma_{k;\mi3}^\eell\sigma_{k;\mi4}^\eell\right] \right. \\
		\left. - \EE\left[\sigma_{j;\mi1}^\eell\sigma_{j;\mi2}^\eell\right]\EE\left[\sigma_{k;\mi3}^\eell\sigma_{k;\mi4}^\eell\right]\right\}
	\end{multline*}
	} 
	\begin{itemize}
		\item Vamos dar um nome para a expressão entre chaves:  $\Xi_{j;k}^\eell$
	\end{itemize}
\end{frame}

\begin{frame}
	Para índices iguais, a equação (4.62) nos dá o seguinte resultado:
	{
\begin{multline*}
	\EE\left[\sigma_{j;\mi1}^\eell\sigma_{j;\mi2}^\eell \sigma_{j;\mi3}^\eell\sigma_{j;\mi4}^\eell\right] - \EE\left[\sigma_{j;\mi1}^\eell\sigma_{j;\mi2}^\eell\right]\EE\left[\sigma_{j;\mi3}^\eell\sigma_{j;\mi4}^\eell\right] = \\
	\Expectation{\sigma_\mi1\sigma_\mi2\sigma_\mi3\sigma_\mi4}{G^\eell} - \Expectation{\sigma_\mi1\sigma_\mi2}{G^\eell}  \Expectation{\sigma_\mi3\sigma_\mi4}{G^\eell}  +  O(\sfrac{1}{n})
\end{multline*}
	}
\end{frame}

\begin{frame}
	Para índices diferentes, a equação (4.63) nos dá o seguinte resultado:
	{\small
	\begin{multline*}
		\EE\left[\sigma_{j;\mi1}^\eell\sigma_{j;\mi2}^\eell \sigma_{k;\mi3}^\eell\sigma_{k;\mi4}^\eell\right] - \EE\left[\sigma_{j;\mi1}^\eell\sigma_{j;\mi2}^\eell\right]\EE\left[\sigma_{k;\mi3}^\eell\sigma_{k;\mi4}^\eell\right] = \\ =
		\frac{1}{4}\sum_{\mj{i}\in\Dcal}^{1\le{i}\le4}v^{(\mj1\mj2)(\mj3\mj4)}_{\eell}\Big(\text{\pmglyph{a}}\Big)  + O(v^2) =\\
		= \frac{1}{4}\sum_{\mj{i}\in\Dcal}^{1\le{i}\le4} \frac{1}{n_\ell}{\Verticeinvb{\ell}1234} \Big(\text{\pmglyph{a}}\Big)  + O(\sfrac{1}{n_\ell^2}) 
	\end{multline*}
	\begin{equation*}
		\text{\pmglyph{a}} = \Expectation{\sigma_{\mi1}\sigma_{\mi2}(z_{mj1}z_{\mj2} - g_{\mj1\mj2}) }{g}\Expectation{\sigma_{\mi3}\sigma_{\mi4}(z_{mj3}z_{\mj4} - g_{\mj3\mj4}) }{g} 
	\end{equation*}
	}
	\begin{itemize} 
		\item O termo \pmglyph{a} é de ordem constante, pois só contém integrais gaussianas dependentes de $G^\eell$.
	\end{itemize}
\end{frame}

\begin{frame}
	Voltando para nossa equação, separamos a soma de índices iguais e diferentes, e aplicamos $n_\ell = n$.
	\small
	\begin{multline*}
		\frac{1}{n}\Vertice{\ell+1}1234 =  \left(\frac{C_W^\eellum}{n}\right)^2 \left\{\sum_{j=k}^{n}\Xi_{j;k}^\eell + \sum_{j\neq k}^{n}\Xi_{j;k}^\eell\right\} = \\
		= \frac{{C_W^\eellum}^2}{n^2} \left\{\sum_{j=1}^{n}\Expectation{\sigma_\mi1\sigma_\mi2\sigma_\mi3\sigma_\mi4}{G^\eell} - \Expectation{\sigma_\mi1\sigma_\mi2}{G^\eell}  \Expectation{\sigma_\mi3\sigma_\mi4}{G^\eell}  +  O(\sfrac{1}{n}) \right.\\ 
		\left. +\sum_{j\neq k}^{n}\left(\frac{1}{4n}\sum_{\mj{i}\in\Dcal}^{1\le{i}\le4}\Verticeinvb{\ell}1234\left(\text{\pmglyph{a}}\right) + O(\sfrac{1}{n^2})\right)\right\} =
	\end{multline*}
\end{frame}

\begin{frame} 
	\small
	\begin{multline*}
		= \frac{{C_W^\eellum}^2}{n^2} \Bigg\{n\big[\Expectation{\sigma_\mi1\sigma_\mi2\sigma_\mi3\sigma_\mi4}{G^\eell} - \Expectation{\sigma_\mi1\sigma_\mi2}{G^\eell}  \Expectation{\sigma_\mi3\sigma_\mi4}{G^\eell}  +  O(\sfrac{1}{n}) \big] \\ 
		+(n^2-n)\left[\frac{1}{4n}\sum_{\mj{i}\in\Dcal}^{1\le{i}\le4}\Verticeinvb{\ell}1234\left(\text{\pmglyph{a}}\right) + O(\sfrac{1}{n^2})\right]\Bigg\} =\\
		= {{C_W^\eellum}^2} \Bigg\{ \frac{1}{n}\big[\Expectation{\sigma_\mi1\sigma_\mi2\sigma_\mi3\sigma_\mi4}{G^\eell} - \Expectation{\sigma_\mi1\sigma_\mi2}{G^\eell}  \Expectation{\sigma_\mi3\sigma_\mi4}{G^\eell} \big]  +  O(\sfrac{1}{n^2})\\
		+ \frac{1}{4n}\left[\sum_{\mj{i}\in\Dcal}^{1\le{i}\le4}\Verticeinvb{\ell}1234\big(\text{\pmglyph{a}}\big)\right] + O(\sfrac{1}{n^2})\Bigg\}
	\end{multline*}
	
\end{frame}

\begin{frame}
	\small
	\begin{multline*}
		 {{C_W^\eellum}^2} \Bigg\{ \frac{1}{n} {\color{blue}
		 \big[\Expectation{\sigma_\mi1\sigma_\mi2\sigma_\mi3\sigma_\mi4}{G^\eell} - \Expectation{\sigma_\mi1\sigma_\mi2}{G^\eell}  \Expectation{\sigma_\mi3\sigma_\mi4}{G^\eell} \big] }  \\
		+ \frac{1}{4n}\left[ {\color{green}\sum_{\mj{i}\in\Dcal}^{1\le{i}\le4}\Verticeinvb{\ell}1234\big(\text{\pmglyph{a}}\big) }\right] \Bigg\} + O(\sfrac{1}{n^2})
	\end{multline*}
	
\end{frame}

\begin{frame}
	Por hipótese de indução, temos que as partes em azul são de ordem constante.
	{\small
	\begin{multline*}
		 \frac{1}{n}\Vertice{\ell+1}1234 = \frac{{C_W^\eellum}^2}{n} \Bigg\{  
		 \left[{\color{blue} \begin{array}{l}
		 \Expectation{\sigma_\mi1\sigma_\mi2\sigma_\mi3\sigma_\mi4}{G^\eell} \\
		 - \Expectation{\sigma_\mi1\sigma_\mi2}{G^\eell}  \Expectation{\sigma_\mi3\sigma_\mi4}{G^\eell} 
		 \end{array} }
\right]   \\
		+ \frac{1}{4}\left[ \sum_{\mj{i}\in\Dcal}^{1\le{i}\le4}{\color{blue}\Verticeinvb{\ell}1234\big(\text{\pmglyph{a}}\big) }\right] \Bigg\} + O(\sfrac{1}{n^2})
	\end{multline*}
	}
	Logo, segue que 
	\begin{equation*}\tag{4.91}
		\frac{1}{n}\Vertice{\ell+1}1234 = O(\sfrac{1}{n}) 
	\end{equation*}
	o que completa a indução. \hfill $\qed$ 
\end{frame}

\section{Marginalization Rules}

\begin{frame}
\tableofcontents[currentsection]
\end{frame}


\begin{frame}
	\begin{itemize}
		\item Queremos calcular o valor esperado de uma função $F(z_{\Ical;\Acal})$, com $\Ical\subset \Ncal =: \{1,\ldots,n_\ell\}$ e $\Acal\subset \Dcal$.
		\item Para conjuntos $X\subset Y$, vamos usar a notação $\overline{X}$ para denotar o conjunto complementar de $X$ em $Y$, notadamente $Y\setminus X$.
		\item Vamos separar as variáveis de integração em dois conjuntos: $\Ical\times\Acal$ que é de interesse e o seu complementar $\overline{\Ical\times\Acal}$.
	\end{itemize}
\end{frame}

\begin{frame}
	\begin{multline*}
		\EE[F(z_{\Ical;\Acal})] = \int\left[{\prod_{(i,\alpha)\in \Ncal\times\Dcal}}\hspace*{-12pt} dz_{i;\alpha}^\eell\right] F(z_{\Ical;\Acal}) p(z_{\Ncal;\Dcal}\mid \Dcal)= \\=
		\int\left[{\prod_{(i,\alpha)\in \Ical\times\Acal}}\hspace*{-12pt} dz_{i;\alpha}^\eell {\prod_{(j,\beta)\in \overline{\Ical\times\Acal}}} \hspace*{-12pt} dz_{j;\beta}^\eell\right] 
		F(z_{\Ical;\Acal}) p(z_{\Ncal;\Dcal}\mid \Dcal)= \\=
		\int\left[{\prod_{(i,\alpha)\in \Ical\times\Acal}} \hspace*{-12pt} dz_{i;\alpha}^\eell\right]F(z_{\Ical;\Acal})\int\left[ {\prod_{(j,\beta)\in \overline{\Ical\times\Acal}}} \hspace*{-12pt} dz_{j;\beta}^\eell\right] 
		 p(z_{\Ncal;\Dcal}\mid \Dcal)= \\
		 =\int\left[{\prod_{(i,\alpha)\in \Ical\times\Acal}}\hspace*{-12pt}  dz_{i;\alpha}^\eell\right]F(z_{\Ical;\Acal})p(z_{\Ical;\Acal}\mid \Acal)\tag{4.92}
	\end{multline*}
\end{frame}

\begin{frame}
	\begin{equation*}\tag{4.93}
		p(z_{\Ical;\Acal}\mid \Acal) := \int\left[{\prod_{(j,\beta)\in \overline{\Ical\times\Acal}}} \hspace*{-12pt} dz_{j;\beta}^\eell\right] p(z_{\Ncal;\Dcal}\mid \Dcal) 
	\end{equation*}
\end{frame}

\subsection{Marginalization over samples}


\begin{frame}
	\begin{itemize}
	% 	\item Vamos considerar o caso em que $\Acal = \{\mi1,\mi2\}$ 
	% {\small
	% \begin{equation*}\tag{4.94}
	% 	\EE[\zia{1}{1}^\eellum\zia{2}{2}^\eellum] = \delta_{i_1i_2}(C_b^\eellum + C_W^\eellum\Expectation{\sigma_{\mi1}\sigma_{\mi2}}{G^\eell} + O(\sfrac{1}{n_\ell})) 
	% \end{equation*} 
	% }
	\item Ao invés de calcular $\Expectation{\sigma_{\mi1}\sigma_{\mi2}}{G^\eell}$ como uma integral $N_\Dcal$-dimensional, podemos calcular como uma integral dupla.
	\item Da mesma forma, podemos calcular $\Expectation{\sigma_{\mi1}\sigma_{\mi2}\sigma_\mi3\sigma_\mi4}{G^\eell}$ como uma integral em no máximo 4 variáveis.
	\item Para o cálculo de $\Vertice{\ell+1}1234$, podemos usar (4.90) e somar somente sobre os índices $\Acal = \{\mi1,\mi2,\mi3,\mi4\}$, lembrando de ajustar a métrica inversa $\Verticeinv{\ell}1234$ 
	\end{itemize}

\end{frame}

\subsection{Marginalization over neurons}

\begin{frame}
	\begin{equation*}\tag{4.95}
		-\frac{1}{8}\sum_{i,j}^{n_\ell} \sum_{\mi{i}\in\Dcal}^{1\le i \le 4} v^{(\mi1\mi2)(\mi3\mi4)}_{\eell} z_{i;\mi1}^\eell z_{i;\mi2}^\eell z_{j;\mi3}^\eell z_{j;\mi4}^\eell \sim O\left(\frac{n_\ell^2}{n_\ell}\right) = O(n)
	\end{equation*}
	\begin{equation*}\tag{4.96}
		\frac{1}{2} \sum_{i}^{n_\ell}\sum_{\mi{i}\in\Dcal}^{1\le i \le 2} g^{\mi1\mi2}_{\eell} z_{i;\mi1}^\eell z_{i;\mi2}^\eell \sim  O(n)
	\end{equation*}

	\begin{itemize}
		\item Como utilizamos $m_\ell \ll n_\ell$, os somatórios reduzem drasticamente o número de termos.
		\item $(4.95) \sim O\left(\frac{m_\ell^2}{n_\ell}\right) = O\left(\frac{1}{n}\right)$
		\item $(4.96) \sim O(m_\ell) = O(1)$
	\end{itemize}
\end{frame}

\subsection{Running couplings with partial marginalizations}

\begin{frame}
	\begin{itemize}
		\item Como estamos calculando apenas sobre um subconjunto de neurônios e amostras, temos que ajustar $g$ e $v$ de acordo.
		\item Para simplificar, vamos considerar apenas um input $x$ e vamos derrubar os índices de amostra.
	\end{itemize}
	\begin{multline*}\tag{4.97}
		p(z_1^\eell, \ldots, z_{m_\ell}^\eell ) \propto e^{-S(z_1^\eell, \ldots, z_{m_\ell}^\eell)}\\
		= \exp\left(-\frac{g_{\eell,m_\ell}}{2}\sum_{i=1}^{m_\ell}  z_i^\eell z_i^\eell + \frac{v_{\eell{\color{gray},m_\ell}}}{8}\sum_{j,k=1}^{m_\ell}  z_j^\eell z_j^\eell z_k^\eell z_k^\eell\right)
	\end{multline*}
\end{frame}

\begin{frame}
	\begin{itemize}
		\item Vamos integrar sobre os últimos $n_\ell - m_\ell$ neurônios, ignorando as constantes de normalização.
	\end{itemize}\small
	\begin{multline*}
		e^{-S(z_1^\eell, \ldots, z_{m_\ell}^\eell)} \propto p(z_1^\eell, \ldots, z_{m_\ell}^\eell ) = \int\Big[\prod_{i=m_\ell+1}^{n_\ell} \hspace*{-10pt}dz_i^\eell\Big] p(z_1^\eell, \ldots, z_{n_\ell}^\eell ) \\ 
		\propto \int\Big[\prod_{i=m_\ell+1}^{n_\ell} \hspace*{-10pt}dz_i^\eell\Big] \exp\left(-\frac{g_{\eell,n_\ell}}{2}\sum_{i=1}^{n_\ell}  z_i^\eell z_i^\eell + \frac{v_{\eell}}{8}\sum_{j,k=1}^{n_\ell}  z_j^\eell z_j^\eell z_k^\eell z_k^\eell\right)\\
	\end{multline*}
\end{frame}

\begin{frame}
	\begin{itemize}
		\item Para simplificar, vamos sumir com os índices $\ell$.
		\item Vamos modificar a notação $$\int \left[\prod_{i=a}^b dz_i\right] = \int_{i=a}^b \hspace{-10pt}dz_i$$
		\item Vamos lembrar que $\exp(a+b) = \exp(a)\exp(b)$.
		\item Vamos separar o somatório duplo $$\sum_{j,k=1}^n = \sum_{j,k=1}^{m} + \sum_{j=1}^{m}\sum_{k=m+1}^n + \sum_{j=m+1}^n\sum_{k=1}^{m} + \sum_{j,k=m+1}^n$$
	\end{itemize}
\end{frame}
\newcommand{\redp}[1]{\textcolor<#1->{red}}
\newcommand{\blup}[1]{\textcolor<#1->{blue}}
\begin{frame}

	\small
	\begin{multline*}
		 p(z_1^\eell, \ldots, z_{m_\ell}^\eell ) \propto \int_{i=m+1}^{n}\hspace{-10pt} dz_i {\redp1\exp}\left[-\frac{g}{2}\sum_{i=1}^{n}  z_i^2 {\redp1+} \frac{v}{8}\sum_{j,k=1}^{n}  z_j^2 z_k^2\right]=\\ % linha 1
		 = \int_{i=m+1}^{n}\hspace{-12pt} dz_i {\redp2\exp}\left[-\frac{g}{2}{\blup3{\sum_{i=1}^{n}}}  z_i^2 \right] {\redp2\exp}\left[ \frac{v}{8}\sum_{j,k=1}^{n}  z_j^2 z_k^2\right]=\\ % linha 2
		 = \int_{i=m+1}^{n}\hspace{-12pt} dz_i {\redp5 \exp}\left[-\frac{g}{2}\sum_{\blup4{i=1}}^{\blup4 m}  z_i^2 {\redp5 -}\frac{g}{2}\sum_{\blup4{i=m+1}}^{\blup4 n}  z_i^2 \right] \exp\left[ \frac{v}{8}\sum_{j,k=1}^{n}  z_j^2 z_k^2\right]=\\ % linha 3
		  = \int_{i=m+1}^{n}\hspace{-12pt} dz_i \blup7{ \exp\left[-\frac{g}{2}\sum_{i=1}^{m}  z_i^2 \right]}{\redp6\exp}\left[ -\frac{g}{2}\sum_{i=m+1}^{n}  z_i^2 \right] \exp\left[ \frac{v}{8}\sum_{j,k=1}^{n}  z_j^2 z_k^2\right]=\\ %linha 4
		   = \blup8{\exp\left[-\frac{g}{2}\sum_{i=1}^{m}  z_i^2 \right]}\int_{i=m+1}^{n}\hspace{-12pt} dz_i \exp\left[ -\frac{g}{2}\sum_{i=m+1}^{n}  z_i^2 \right] \exp\left[ \frac{v}{8}\redp9{\sum_{j,k=1}^{n}  z_j^2 z_k^2}\right]=\\ 
	\end{multline*}
\end{frame}

\begin{frame}
	\small
	\begin{multline*}
		= \exp\left[-\frac{g}{2}\sum_{i=1}^{m}  z_i^2 \right]\int_{i=m+1}^{n}\hspace{-12pt} dz_i \exp\left[ -\frac{g}{2}\sum_{i=m+1}^{n}  z_i^2 \right] \exp\left[ \redp2{\frac{v}{8}\sum_{j,k=1}^{m}  z_j^2z_k^2 +} \right.\\ % linha 1
		\left. \blup4{ + \frac{v}{8}\sum_{j=1}^{m}\sum_{k=m+1}^{n} \hspace*{-6pt}z_j^2z_k^2 +\frac{v}{8}\sum_{j=m+1}^{n}\sum_{k=1}^{n}z_j^2z_k^2  }+ \frac{v}{8}\sum_{j,k=m+1}^{n}  z_j^2 z_k^2\right]=\\ % linha 2
		= \exp\left[-\frac{g}{2} \sum_{i=1}^{m}  z_i^2 \right] \redp3{\exp\left[ \frac{v}{8}\sum_{j,k=1}^{m}  z_j^2z_k^2 \right]}\times \\
		\times \int_{i=m+1}^{n}\hspace{-12pt} dz_i \exp\left[ -\frac{g}{2}\sum_{i=m+1}^{n}  z_i^2 \right] 
		\exp\left[ \frac{v}{8}\left(\blup5{2\sum_{j=1}^{m}\sum_{k=m+1}^{n} \hspace*{-6pt}z_j^2z_k^2 }+ \hspace{-6pt}\sum_{j,k=m+1}^{n} \hspace{-6pt} z_j^2 z_k^2  \right)  \right]
	\end{multline*}
\end{frame}

\begin{frame}
	\begin{itemize}
		\item Agora usamos $\exp(v\Sigma) \approx 1 + v\Sigma + O(v^2)$ para trocar a expoencial
	\end{itemize}
	\begin{multline*}
		\exp\left[ \frac{v}{8}\left(2\sum_{j=1}^{m}\sum_{k=m+1}^{n} \hspace*{-6pt}z_j^2z_k^2 + \hspace{-6pt}\sum_{j,k=m+1}^{n} \hspace{-6pt} z_j^2 z_k^2  \right)  \right] \approx\\
		\approx 1 + \frac{2v}{8}\sum_{j=1}^{m}\sum_{k=m+1}^{n} \hspace*{-6pt}z_j^2z_k^2 +  \frac{v}{8}\hspace{-6pt}\sum_{j,k=m+1}^{n} \hspace{-6pt} z_j^2 z_k^2   + O(v^2)
	\end{multline*}
\end{frame}

\begin{frame}

	{\footnotesize
	\begin{multline*}
		p(z_1^\eell, \ldots, z_{m_\ell}^\eell ) \propto \exp\left[-\frac{g}{2}\sum_{i=1}^{m}  z_i^2  + \frac{v}{8}\sum_{j,k=1}^{m}  z_j^2z_k^2 + \right]\times\\
		\times \int_{i=m+1}^{n}\hspace{-12pt} dz_i \exp\left[ -\frac{g}{2}\sum_{i=m+1}^{n}  z_i^2 \right] 
		\left(1+ \underbrace{\frac{2v}{8}\sum_{j=1}^{m}\sum_{k=m+1}^{n} \hspace*{-6pt}z_j^2z_k^2}_{(II)} +  \underbrace{\frac{v}{8}\hspace{-6pt}\sum_{j,k=m+1}^{n} \hspace{-6pt} z_j^2 z_k^2}_{(III)}  + O(v^2)\right)
	\end{multline*}}
	Vamos aplicar a propriedade distribuitiva e resolver as integrais $(II)$ e $(III)$ separadamente.
\end{frame}

\begin{frame}

	\small
	\begin{multline*}
		(II) = \int_{i=m+1}^{n}\hspace{-12pt} dz_i \exp\left[ -\frac{g}{2}\sum_{i=m+1}^{n}  z_i^2 \right] \frac{2v}{8}\sum_{j=1}^{m}\sum_{k=m+1}^{n} \hspace*{-6pt}z_j^2z_k^2 = \\ % linha 1
		 \frac{v}{4}\int_{i=m+1}^{n}\hspace{-12pt} dz_i \exp\left[ -\frac{g}{2}\sum_{i=m+1}^{n}  z_i^2 \right]\left[\sum_{j=1}^{m}z_j^2\right]\left[\sum_{k=m+1}^{n}z_k^2\right] \\ % linha 2
		 \frac{v}{4}\left[\sum_{j=1}^{m}z_j^2\right]\underbrace{\int_{i=m+1}^{n}\hspace{-12pt} dz_i \exp\left[ -\frac{g}{2}\sum_{i=m+1}^{n}  z_i^2 \right]\left[\sum_{k=m+1}^{n}z_k^2\right]}_{(ii)}
	\end{multline*}
\end{frame}


\begin{frame}
	Para resolver $(ii)$, vamos usar a linearidade da exponencial:	{\small
	\begin{multline*}
		(ii) = \int_{i=m+1}^{n}\hspace{-12pt} dz_i \exp\left[ -\frac{g}{2}\sum_{i=m+1}^{n}  z_i^2 \right]\left[\sum_{k=m+1}^{n}z_k^2\right]  = \\ % linha 1
		\sum_{k=m+1}^{n} \int_{i=m+1}^{n}\hspace{-12pt} dz_i \exp\left[ -\frac{g}{2}\sum_{i=m+1}^{n}  z_i^2 \right] z_k^2 =: 
		\sum_{k=m+1}^{n} I_k\\ % linha 2
	\end{multline*}}
	Vamos usar a propriedade da exponencial e o teorema de Fubini para calcular $I_k$.
\end{frame}

\begin{frame}
	
	{\small
	\begin{multline*}
		I_k = \int_{i=m+1}^{n}\hspace{-12pt} dz_i \exp\left[ -\frac{g}{2}\sum_{i=m+1}^{n}  z_i^2 \right] z_k^2 = \int_{i=m+1}^{n}\hspace{-12pt} dz_i \prod_{i=m+1}^{n}\exp\left[ -\frac{g}{2}  z_i^2 \right] z_k^2 = \\
		= \int dz_k \ z_k^2 \exp\left[-\frac{g}{2}z_k^2\right] \prod_{\stackrel{i=m+1}{i\neq k}}^{n}\int dz_i\exp\left[ -\frac{g}{2}  z_i^2 \right] = \\ 
		\frac{\sqrt{\pi}}{2g^{3/2}} \cdot \text{Constante} = \frac{1}{g}\frac{C_1}{\sqrt{g}} \\
	\end{multline*}
	}
	Voltando para $(ii)$, temos que
	\begin{equation*}
		(ii) = \sum_{k=m+1}^{n} I_k = (n-m)I_k = \frac{(n-m)}{g}\frac{C_1}{\sqrt{g}}  
	\end{equation*}
\end{frame}

\begin{frame}
	Voltando para $(II)$, temos que
	\begin{multline*}
		(II) = \frac{v}{4}\left[\sum_{j=1}^{m}z_j^2\right]\left[\sum_{k=m+1}^{n}I_k\right] = \frac{v}{4}\left[\sum_{j=1}^{m}z_j^2\right]\left[\frac{(n-m)}{g}\frac{C_1}{\sqrt{g}}\right] = \\ 
		= \frac{(n-m)}{4}\frac{v}{h}\frac{C_1}{\sqrt{g}}\left[\sum_{j=1}^{m}z_j^2\right]
	\end{multline*}

	Vamos calcular $(III)$ agora, separando o somatório de índices iguais e diferentes.
\end{frame}

\begin{frame}

	\small
	\begin{multline*}
		(III) = \int_{i=m+1}^{n}\hspace{-12pt} dz_i \exp\left[ -\frac{g}{2}\sum_{i=m+1}^{n}  z_i^2 \right] \frac{v}{8}\hspace{-6pt}\sum_{j,k=m+1}^{n} \hspace{-6pt} z_j^2 z_k^2 = \\ % linha 1
		= \frac{v}{8}\int_{i=m+1}^{n}\hspace{-12pt} dz_i \exp\left[ -\frac{g}{2}\sum_{i=m+1}^{n}  z_i^2 \right]\left[\sum_{\nu=m+1}^{n}z_\nu^4 + \sum_{\stackrel{\mu,\nu=m+1}{\mu\neq\nu}}^{n}z_\nu^2z_\mu^2\right]  \\ % linha 2 
		= \frac{v}{8}\left[\sum_{\nu=m+1}^{n} \underbrace{\int_{i=m+1}^{n}\hspace{-12pt} dz_i\exp\left[ -\frac{g}{2}\sum_{i=m+1}^{n}  z_i^2 \right]z_\nu^4}_{(iii.1)}\right. + \\ 
		\left.\sum_{\stackrel{\mu,\nu=m+1}{\mu\neq\nu}}^{n}\underbrace{\int_{i=m+1}^{n}\hspace{-12pt} dz_i \exp\left[ -\frac{g}{2}\sum_{i=m+1}^{n}  z_i^2 \right]z_\nu^2z_\mu^2}_{(iii.2)}\right]  % linha 3
	\end{multline*}
\end{frame}

\begin{frame}
	Para resolver $(iii.1)$, vamos usar a propriedade da exponencial e o teorema de Fubini:
	{\small
	\begin{multline*}
		(iii.1) = \int_{i=m+1}^{n}\hspace{-12pt} dz_i \exp\left[ -\frac{g}{2}\sum_{i=m+1}^{n}  z_i^2 \right]z_\nu^4 = \int_{i=m+1}^{n}\hspace{-12pt} dz_i \prod_{i=m+1}^{n}\exp\left[ -\frac{g}{2}  z_i^2 \right]z_\nu^4 = \\ 
		= \int dz_\nu z_\nu^4 \exp\left[-\frac{g}{2}z_\nu^2\right] \prod_{\stackrel{i=m+1}{i\neq\nu}}^{n}\int dz_i\exp\left[ -\frac{g}{2}  z_i^2 \right] = \\  
		=\frac{3\sqrt{\pi}}{4g^{5/2}} \cdot \text{Constante} = \frac{1}{g^2}\frac{C_2}{\sqrt{g}} 
	\end{multline*}}
	Vamos calcular $(iii.2)$ agora.
\end{frame}

\begin{frame}
	Para resolver $(iii.2)$, vamos usar a propriedade da exponencial e o teorema de Fubini:
	{\small
	\begin{multline*}
		(iii.2) = \int_{i=m+1}^{n}\hspace{-12pt} dz_i \exp\left[ -\frac{g}{2}\sum_{i=m+1}^{n}  z_i^2 \right]z_\nu^2z_\mu^2 = \int_{i=m+1}^{n}\hspace{-12pt} dz_i \prod_{i=m+1}^{n}\exp\left[ -\frac{g}{2}  z_i^2 \right]z_\nu^2z_\mu^2 = \\ 
		= \int dz_\nu z_\nu^2 \exp\left[-\frac{g}{2}z_\nu^2\right] \int dz_\mu z_\mu^2 \exp\left[-\frac{g}{2}z_\mu^2\right] \prod_{\stackrel{i=m+1}{i\neq\nu,\mu}}^{n}\int dz_i\exp\left[ -\frac{g}{2}  z_i^2 \right] = \\  
		=\frac{\sqrt{\pi}}{2g^{3/2}}\frac{\sqrt{\pi}}{2g^{3/2}} \cdot \text{Constante} = \frac{1}{g^3}C_3
	\end{multline*}}
	Vamos voltar para $(III)$ agora.
\end{frame}

\begin{frame}
	Voltando para $(III)$, temos que
	{\small
	\begin{multline*}
		(III) = \frac{v}{8}\left[\sum_{\nu=m+1}^{n} (iii.1)\right] + \frac{v}{8}\left[\sum_{\stackrel{\mu,\nu=m+1}{\mu\neq\nu}}^{n} (iii.2)\right] = \\ 
		= \frac{v}{8}\left[\sum_{\nu=m+1}^{n} \frac{1}{g^2}\frac{C_2}{\sqrt{g}}\right] + \frac{v}{8}\left[\sum_{\stackrel{\mu,\nu=m+1}{\mu\neq\nu}}^{n} \frac{1}{g^3}C_3\right] = \\ 
		= \frac{v}{8}\left[\frac{(n-m)}{g^2}\frac{C_2}{\sqrt{g}} + \frac{(n-m)^2 - (n-m)}{g^3}C_3\right]
	\end{multline*}}
	Agora, vamos juntar os resultados de $(II)$ e $(III)$.
\end{frame}

\begin{frame}
	\footnotesize
	\begin{multline*}
		\exp\left[-\frac{g}{2}\sum_{i=1}^{m}  z_i^2  + \frac{v}{8}\sum_{j,k=1}^{m}  z_j^2z_k^2 + \right]\times\\ % linha 1
		\times \int_{i=m+1}^{n}\hspace{-12pt} dz_i \exp\left[ -\frac{g}{2}\sum_{i=m+1}^{n}  z_i^2 \right] 
		\left(1+ \frac{2v}{8}\sum_{j=1}^{m}\sum_{k=m+1}^{n} \hspace*{-6pt}z_j^2z_k^2 +  \frac{v}{8}\hspace{-6pt}\sum_{j,k=m+1}^{n} \hspace{-6pt} z_j^2 z_k^2 + O(v^2)\right) \\ % linha 2
	= \exp\left[-\frac{g}{2}\sum_{i=1}^{m}  z_i^2  + \frac{v}{8}\sum_{j,k=1}^{m}  z_j^2z_k^2 + \right]\times\\
	\times\left\{1+\frac{(n-m)}{4}\frac{v}{h}\frac{C_1}{\sqrt{g}}\left[\sum_{j=1}^{m}z_j^2\right] + \frac{v}{8}\left[\frac{(n-m)}{g^2}\frac{C_2}{\sqrt{g}} + \frac{(n-m)^2 - (n-m)}{g^3}C_3\right] + O(v^2)\right\} \\ % linha 3
\textcolor<1->{gray}{\times \left\{1+\frac{(n-m)}{4}\frac{v}{g}\left[\sum_{j=1}^{m}z_j^2\right] + \frac{v}{8g^2}\left[ (n-m)^2 +2(n-m)\right] + O(v^2)\right\} }
	\end{multline*}
\end{frame}

\begin{frame}
	Utilizando umas aproximações bem aproximadas, podemos escrever
	\footnotesize
	\begin{multline*}
		\exp\left[-\frac{g_{\eell,n_\ell}}{2}\sum_{i=1}^{m}  z_i^2   \blup{2}{+ \frac{v_\eell}{8}\sum_{j,k=1}^{m}  z_j^2z_k^2 } \right]\times\\ % linha 1
		\times \left\{1+\frac{(n_\ell-m_\ell)}{4}\frac{v_\eell}{g_{\eell,n_\ell}}\left[\sum_{j=1}^{m}z_j^2\right] \blup{2}{+ \frac{v_\eell}{8g_{\eell,n_\ell}^2}\left[ (n-m)^2 +2(n-m)\right] + O(v^2)}\right\}\\
		\approx \exp\left[-\frac{g_{\eell,n_\ell}}{2}\sum_{i=1}^{m}  z_i^2 + \ldots \right]\times \left\{1+\frac{(n_\ell-m_\ell)}{4}\frac{v_\eell}{g_{\eell,n_\ell}}\left[\sum_{j=1}^{m}z_j^2\right] + \ldots  \right\}\\
		\approx \exp\left[-\frac{g_{\eell,n_\ell}}{2}\sum_{i=1}^{m}  z_i^2 \ldots \right]\exp \left[\frac{(n_\ell-m_\ell)}{4}\frac{v_\eell}{g_{\eell,n_\ell}}\left[\sum_{j=1}^{m}z_j^2\right] \right]\\
		= \exp\left[\left(-\frac{g_{\eell,n_\ell}}{2} + \frac{(n_\ell-m_\ell)}{4}\frac{v_\eell}{g_{\eell,n_\ell}}\right)\sum_{j=1}^{m}z_j^2 \right]
	\end{multline*}
\end{frame}


\begin{frame}
	Comparando o termo da expressão acima que multiplica $\sum_{j=1}^{m}z_j^2$ com o mesmo termo equação (4.97), temos que 
	\[
	-\frac{g_{\eell,n_\ell}}{2} + \frac{(n_\ell-m_\ell)}{4}\frac{v_\eell}{g_{\eell,n_\ell}} = -\frac{g_{\eell,m_\ell}}{2}
	\]
	o que nos leva a
	\begin{equation*}\tag{4.100}
		g_{\eell,m_\ell} = g_{\eell,n_\ell} - \frac{(n_\ell -m_\ell)}{4}\frac{v_{\eell}}{g_{\eell,n_\ell}} + O(v^2)
	\end{equation*}  
\end{frame}


\begin{frame}
	\begin{equation*}\tag{4.101}
		G^\eell = \frac{1}{g_{\eell,m_\ell}} + \frac{(m_\ell+2)}{2}\frac{v^\eell}{g^3_{\eell,m_\ell}} + O\left(v^2\right)
	\end{equation*}

	\begin{equation*}\tag{4.102}
		\frac{1}{g_{\eell,m_\ell}} = G^\eell - \frac{(m_\ell+2)}{2}\frac{V^\eell}{n_{\ell-1}G ^\eell} + O\left(\frac{1}{n^2}\right)
	\end{equation*}

	\begin{equation*}
		v^\eell = \frac{V^\eell}{n_{\ell-1}(G^\eell)^4 } + O\left(\frac{1}{n^2}\right)  
	\end{equation*}
\end{frame}

\begin{frame}
	\small
	\begin{multline*}\tag{4.47}
		\EE[\zia11\zia22] = \delta_{i_1 i_2}\times\\\times\left[g_{\mi1\mi2} + \frac{1}{2} \sum_{\mj{i}\in\Dcal}^{1\le{i}\le4} v^{(\mj1\mj2)(\mj3\mj4)}(ng_{\mi1\mj1}g_{\mi2\mj2}g_{\mj3\mj4} +2g_{\mi1\mj1}g_{\mi2\mj3}g_{\mj2\mj4})\right]
	\end{multline*}
	 
\end{frame}

\end{document}

