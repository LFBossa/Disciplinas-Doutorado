% Options for packages loaded elsewhere
\PassOptionsToPackage{unicode}{hyperref}
\PassOptionsToPackage{hyphens}{url}
%
\documentclass[
  ignorenonframetext,
  aspectratio=169,
]{beamer}
\usepackage{pgfpages}
\setbeamertemplate{caption}[numbered]
\setbeamertemplate{caption label separator}{: }
\setbeamercolor{caption name}{fg=normal text.fg}
\beamertemplatenavigationsymbolsempty
% Prevent slide breaks in the middle of a paragraph
\widowpenalties 1 10000
\raggedbottom
\setbeamertemplate{part page}{
  \centering
  \begin{beamercolorbox}[sep=16pt,center]{part title}
    \usebeamerfont{part title}\insertpart\par
  \end{beamercolorbox}
}
\setbeamertemplate{section page}{
  \centering
  \begin{beamercolorbox}[sep=12pt,center]{part title}
    \usebeamerfont{section title}\insertsection\par
  \end{beamercolorbox}
}
\setbeamertemplate{subsection page}{
  \centering
  \begin{beamercolorbox}[sep=8pt,center]{part title}
    \usebeamerfont{subsection title}\insertsubsection\par
  \end{beamercolorbox}
}
\AtBeginPart{
  \frame{\partpage}
}
\AtBeginSection{
  \ifbibliography
  \else
    \frame{\sectionpage}
  \fi
}
\AtBeginSubsection{
  \frame{\subsectionpage}
}
\usepackage{amsmath,amssymb}
\usepackage{iftex}
\ifPDFTeX
  \usepackage[T1]{fontenc}
  \usepackage[utf8]{inputenc}
  \usepackage{textcomp} % provide euro and other symbols
\else % if luatex or xetex
  \usepackage{unicode-math} % this also loads fontspec
  \defaultfontfeatures{Scale=MatchLowercase}
  \defaultfontfeatures[\rmfamily]{Ligatures=TeX,Scale=1}
\fi
\usepackage{lmodern}
\usetheme[]{Warsaw}
\usecolortheme{beaver}
\ifPDFTeX\else
  % xetex/luatex font selection
\fi
% Use upquote if available, for straight quotes in verbatim environments
\IfFileExists{upquote.sty}{\usepackage{upquote}}{}
\IfFileExists{microtype.sty}{% use microtype if available
  \usepackage[]{microtype}
  \UseMicrotypeSet[protrusion]{basicmath} % disable protrusion for tt fonts
}{}
\makeatletter
\@ifundefined{KOMAClassName}{% if non-KOMA class
  \IfFileExists{parskip.sty}{%
    \usepackage{parskip}
  }{% else
    \setlength{\parindent}{0pt}
    \setlength{\parskip}{6pt plus 2pt minus 1pt}}
}{% if KOMA class
  \KOMAoptions{parskip=half}}
\makeatother
\usepackage{xcolor}
\newif\ifbibliography
\setlength{\emergencystretch}{3em} % prevent overfull lines
\providecommand{\tightlist}{%
  \setlength{\itemsep}{0pt}\setlength{\parskip}{0pt}}
\setcounter{secnumdepth}{-\maxdimen} % remove section numbering
\ifLuaTeX
  \usepackage{selnolig}  % disable illegal ligatures
\fi
\IfFileExists{bookmark.sty}{\usepackage{bookmark}}{\usepackage{hyperref}}
\IfFileExists{xurl.sty}{\usepackage{xurl}}{} % add URL line breaks if available
\urlstyle{same}
\hypersetup{
  pdftitle={Distribuições Multivariadas},
  pdfauthor={L. F. Bossa},
  hidelinks,
  pdfcreator={LaTeX via pandoc}}
\usepackage[brazil]{babel}
\title{Distribuições Multivariadas}
\author{L. F. Bossa}
\date{24/03/25}
\institute{UFSC}


\newcommand{\PP}{\mathbb{P}}
\newcommand{\EE}{\mathbb{E}}
\newcommand{\RR}{\mathbb{R}}
\begin{document}

\frame{\titlepage}

\section{Variável Aleatória unidimensional}

\begin{frame}{Variável Aleatória unidimensional}

Uma variável aleatória $X$ é uma função que associa um número real a cada resultado de um experimento aleatório.
Podemos caracterizar uma variável aleatória por sua função densidade de probabilidade $f(x)$, que satisfaz a propriedade
\[\PP(X \le a) = \int_{-\infty}^{a} f(x) dx \] 
\end{frame}

\begin{frame}{Esperança}
Também conhecido como média e denotado por \(\mu\).

\[\mathbb{E}(X) = \int_{\RR} xf(x) dx\]
\end{frame}

\begin{frame}
Em geral, dada \(g: \RR\to \RR\) temos

\[\mathbb{E}(g(X)) = \int_{\RR} g(x)f(x) dx\]
\end{frame}

\begin{frame}{Variância}
\textbf{Variância}

\[\text{Var}(X) = \mathbb{E}( (X - \mu)^2)\]

Mede o quão dispersos estão os dados de \(X\): quanto maior a variância,
mais longe da média estão os dados.

A raiz quadrada da variância é o desvio-padrão, geralmente denotado por
\(\sigma\).
\end{frame}

\begin{frame}{Variáveis Gaussianas}
São caracterizadas pela sua função de densidade

\[f(x) = \frac{1}{\sqrt{2\pi}\sigma}\exp\left(-\frac{(x-\mu)^2}{2\sigma}\right)\]

Uma variável aleatória \(X\) com média \(\mu\) e variância \(\sigma^2\)
é denotada por \(X\sim N(\mu,\sigma^2)\).
\end{frame}

\begin{frame}{Z-score}
Dada uma variável aleatória \(X\sim N(\mu,\sigma^2)\), podemos fazer uma
transformação

\[Z = \frac{X- \mu}{\sigma}\]

de modo que \(Z \sim  N(0,1)\).
\end{frame}

\section{Variável Aleatória em várias dimensões}

\begin{frame}{Motivação}
\begin{itemize}
\tightlist
\item
  Dificilmente uma coleta de dados vai coletar apenas um dado de cada
  amostra estudada
\item
  Fazer uma análise conjunta dos dados permite usar ferramentas de
  álgebra linear (análise de componentes principais, clustering)
\end{itemize}
\end{frame}

\begin{frame}
\begin{itemize}
\tightlist
\item
  Temos um vetor \(X = (X_1, \ldots, X_n)\) cujas componentes são
  variáveis aleatórias.
\item
  Temos uma função de densidade \(f(x_1,\ldots, x_n)\) de modo que
\end{itemize}

\[\PP(X_1 \le a_a, \ldots, X_n \le a_n) = \int_{-\infty}^{a_1}\ldots\int_{-\infty}^{a_n} f(x_1,\ldots,x_n) dx_n\ldots dx_1 \]
\end{frame}

\begin{frame}{Probabilidade marginal}
Surge quando queremos estudar a distribuição geral de uma variável só,
"ignorando" as outras. Nesse caso, integramos a variável de interesse no
intervalo de interesse, e as outras variáveis são integradas em todo
\(\RR\).

\[\PP(X_1 \le a_1)  = \int_{-\infty}^{a_1}\ldots\int_{-\infty}^{\infty} f(x_1,\ldots,x_n) dx_n\ldots dx_1\]

Podemos definir então as densidades marginais

\[f_1(x) =  \int_{-\infty}^{x}\ldots\int_{-\infty}^{\infty} f(x_1,\ldots,x_n) dx_n\ldots dx_1\]
\end{frame}

\begin{frame}{Esperança}
No caso vetorial, o operador esperança atua entrada-a-entrada

\[\EE(X) = (\EE(X_1), \EE(X_2),\ldots, \EE(X_n))\]

em que cada esperança é calculada com respeito à distribuição marginal
de cada variável.
\end{frame}

\begin{frame}
\textbf{Propriedades}

\[\EE(c^\top X) = c^\top\mu\]
\end{frame}

\begin{frame}{Covariância}
Dadas variáveis \(X\), \(Y\) unidimensionais com médias
\(\mu_X, \mu_Y\), definimos a covariância de \(X\) e \(Y\) como
\[\text{Cov}(X,Y) = \EE( (X - \mu_X)(Y - \mu_Y))\]

Por vezes também é usada a notação \(\sigma_{XY}\) para denotar a
covariância.

A covariância indica o quanto \(X\) e \(Y\) estão \emph{linearmente}
relacionadas.
\end{frame}

\begin{frame}
Note que
\[\text{Cov}(X,X) = \EE( (X - \mu_X)(X - \mu_X)) =  \EE((X-\mu_X)^2) =  \text{Var}(X)\]
\end{frame}

\begin{frame}{Independência estatística}
Duas variáveis \(X_1\) e \(X_2\) são ditas independentes se for possível
escrever a função de densidade como \(f(x_1,x_2) = f_1(x_1)f_2(x_2)\)

\begin{block}{Teorema}
Se \(X_1\) e \(X_2\) são independentes, então \(\text{Cov}(X,Y)=0\)
\end{block}
\end{frame}

\begin{frame}
\begin{block}{Contra-exemplo}
A recíproca não é verdadeira. Considere \(X\) uma variável aleatória com
média zero e densidade \(f(x)\) sendo uma função par. Defina
\(Y = X^2\). Nesse caso, claramente \(Y\) e \(X\) não são independentes,
mas \[\text{Cov}(X,Y) = 0 \]
\end{block}
\end{frame}

\begin{frame}{Correlação}
É a covariância "normalizada"

\[\text{Corr}(X,Y) = \frac{\text{Cov}(X,Y)}{\sigma_X\cdot\sigma_Y}\]

Também denotada por \(\rho_{XY}\). Por quê é normalizada?
\end{frame}

\begin{frame}{Matriz de Variância-Covariância}
\begin{itemize}
\tightlist
\item
  Para calcular a variância de uma variável unidimensional, temos que
  calcular \(\EE((X-\mu_X)^2)\)
\item
  Por causa do termo quadrático, temos que adaptar esse cálculo para o
  caso vetorial.
\end{itemize}

Sendo \(X = (X_1,\ldots,X_n)\) uma variável aleatória e seja
\(\EE(X) = \mu_X\) seu vetor-médio. Calculamos a variância de
\(X\) como

\[\Sigma = \EE((X-\mu_X)(X-\mu_X)^\top)\]

\[Var(c^\top X) = c^\top\Sigma c\]
\end{frame}

\begin{frame}
\[\Sigma = \EE\begin{pmatrix}  
(X_1 -\mu_{X_1})(X_1 -\mu_{X_1}) & (X_1 -\mu_{X_1})(X_2 -\mu_{X_2}) & \ldots &  (X_1 -\mu_{X_1})(X_n -\mu_{X_n}) \\
(X_2 -\mu_{X_2})(X_1 -\mu_{X_1}) & (X_2 -\mu_{X_2})(X_2 -\mu_{X_2}) & \ldots &  (X_2 -\mu_{X_2})(X_n -\mu_{X_n}) \\
\vdots & \vdots  & \ddots & \vdots \\
(X_n -\mu_{X_n})(X_1 -\mu_{X_1}) & (X_n -\mu_{X_n})(X_2 -\mu_{X_2}) & \ldots & (X_n -\mu_{X_n})(X_n -\mu_{X_n})
\end{pmatrix}\]
\end{frame}

\begin{frame}
\[\Sigma = \begin{pmatrix}  
\text{Var}(X_1) & \text{Cov}(X_1,X_2) & \ldots & \text{Cov}(X_1,X_n)\\
\text{Cov}(X_1,X_2) & \text{Var}(X_2) & \ldots & \text{Cov}(X_2,X_n)\\
\vdots & \vdots  & \ddots & \vdots \\
\text{Cov}(X_n,X_1) & \text{Cov}(X_n,X_2) & \ldots & \text{Var}(X_n)\\
\end{pmatrix}\]
\end{frame}

\begin{frame}
\begin{block}{Teorema}
A matriz de variância-covariância é definida positiva.
\end{block}

Dado um vetor \(w\) qualquer, note que

\[w^\top \Sigma w = w^\top\EE(XX^\top)w = \EE((Xw)^\top Xw)\]
\end{frame}

\begin{frame}{Como resumir a variância?}
Por vezes, não queremos apresentar a variância como uma matriz, mas como
um número que resuma o quão dispersos estão nossos dados.

Para isso, temos a

\begin{block}{Variância Total}
\[\text{tr}(\Sigma) = \sum_{i=1}^n \text{Var}(X_i)\]
\end{block}

\begin{block}{Variância Generalizada}
\[\text{det}(\Sigma)\]
\end{block}
\end{frame}

\begin{frame}{Distância Mahalanobis}
Dados vetores \(X\) e \(Y\) que pertençam a uma distribuição
multivariada descrita por uma matriz \(\Sigma\), definimos a distância
Mahalanobi entre eles como

\[d_{XY}^2 = (X-Y)^\top\Sigma^{-1}(X-Y)\]

Essa é a generalização do Z-score.
\end{frame}

\begin{frame}{Variáveis gaussianas multivariadas}
Finalmente estamos aptos a entender a fórmula para a distribuição
gaussiana multivariada.

Para tal, vamos reinterpretar a fórmula da gaussiana em 1 dimensão
\end{frame}

\begin{frame}
A gaussiana padrão é dada por

\[f(x) = \frac{1}{\sqrt{2\pi}}\exp\left(-\frac{1}{2}x^2\right)\]
\end{frame}

\begin{frame}
Podemos aplicar uma mudança de variáveis para alterar seu centro e
dispersão, lembrando fazer uma escala para que a integral sobre
\(\mathbb
{R}\) continue sendo \(1\).

\[\frac{1}{\sigma}f\left(\frac{x-\mu}{\sigma}\right) = \frac{1}{\sqrt{2\pi}\sigma}\exp\left(-\frac{1}{2}\left(\frac{x-\mu}{\sigma}\right)^2\right)\]
\end{frame}

\begin{frame}
Mas note que termos quadráticos não generalizam bem para vetores. Para
tanto, vamos reescrever o lado fireito como

\[\frac{1}{\sqrt{2\pi\sigma^2}}\exp\left(-\frac{1}{2}\frac{(x-\mu)^2}{\sigma^2}\right) = \frac{1}{\sqrt{2\pi\sigma^2}}\exp\left(-\frac{1}{2}(x-\mu)(\sigma^2)^{-1}(x-\mu)\right) \]
\end{frame}

\begin{frame}
Nessa expressão fica mais claro que podemos substituir \(\sigma^2\) pela
matriz de covariância, \(x-\mu\) por vetores, e o termo de escala por um
determinante.

Nesse caso, temos a distribuição gaussiana multivariada com média
\(\vec \mu\) e matriz de variância-covariância \(\Sigma\), cuja
expressão é dada por

\[f(X) = \frac{1}{\sqrt{|2\pi\Sigma|}}\exp\left(-\frac{1}{2}\left(X-\vec\mu\right)\Sigma^{-1}(X-\vec\mu)\right) \]

\begin{itemize}
\tightlist
\item
  falar um pouco de principal component analisys
\end{itemize}
\end{frame} 
 

\end{document}
